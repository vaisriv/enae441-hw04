\input{$UNI_DIR/msc/tex/HWSetup}
\input{$UNI_DIR/msc/tex/EngBindings}

%
% Homework Details
%   - Title
%   - Subtitle
%   - Due date
%   - Due time
%   - Course
%   - Section/Time
%   - Instructor
%   - Author
%

\newcommand{\hmwkTitle}{HW04}
\newcommand{\hmwkSubTitle}{Random Variables}
\newcommand{\hmwkDueDate}{November 11th. 2025}
\newcommand{\hmwkDueTime}{09:30 AM}
\newcommand{\hmwkClass}{ENAE 441 - 0101}
\newcommand{\hmwkClassTime}{09:30 AM}
\newcommand{\hmwkClassInstructor}{Dr. Martin}
\newcommand{\hmwkAuthorName}{\textbf{Vai Srivastava}}
\newcommand{\hmwkCompletionDate}{\today}

\begin{document}

\maketitle

\pagebreak

\begin{hwkProblem}{1}{Standard Deviation} \label{hwk:p01}

	Consider two zero-mean uncorrelated random variables \(W\) and \(V\) with standard deviations \(\sigma_w\) and \(\sigma_v\) respectively. What is the standard deviation of the random variable \(X=W+V\)? Note: Do not assume gaussian distribution.

	\hwkSol{} \label{hwk:s01}

	If \(W, V\) are zero-mean and uncorrelated, then
	\[
		\Var[W+V]=\Var[W]+\Var[V]+2 \Cov[W, V]=\sigma_w^2+\sigma_v^2
	\]
	Hence
	\[
		\boxed{\sigma_X=\sqrt{\sigma_w^2+\sigma_v^2}}
	\]

\end{hwkProblem}

\begin{hwkProblem}{2}{Correlation Coefficient} \label{hwk:p02}

	Consider two scalar RVs \( X \) and \( Y \).

	\begin{enumerate}[label=(\alph*)]
		\item \label{hwk:p02a} Prove that if \( X \) and \( Y \) are independent, then their correlation coefficient \( \rho = 0 \).
		\item \label{hwk:p02b} Find an example of two RVs that are not independent but that have a correlation coefficient of zero.
		\item \label{hwk:p02c} Prove that if \( Y \) is a linear function of \( X \) then \( \rho = \pm 1 \).
	\end{enumerate}

	\hwkSol{} \label{hwk:s02}

	\hwkPart{} \label{hwk:s02a}

	If \(X, Y\) are independent and have finite second moments, then
	\[
		\Cov[X, Y]=\mathbb{E}[X Y]-\mathbb{E}[X] \mathbb{E}[Y]=0,
	\]
	so
	\[
		\boxed{\rho=\frac{\Cov[X, Y]}{\sigma_X \sigma_Y}=0}
	\]
	(provided \(\left.\sigma_X, \sigma_Y>0\right)\)

	\hwkPart{} \label{hwk:s02b}

	Let \(X \sim \fn{Unif}[-1,1]\) and \(Y=X^2\). Then

	\[
		\mathbb{E}[X]=0, \quad \mathbb{E}[Y]=\frac{1}{3}, \quad \mathbb{E}[X Y]=\mathbb{E}\left[X^3\right]=0,
	\]

	so \(\Cov[X, Y]=0\) and \(\rho=0\), yet \(Y\) is a deterministic function of \(X \implies\) not independent.

	\hwkPart{} \label{hwk:s02c}

	If \(Y=a X+b\) with \(a \neq 0\),
	\[
		\Cov[X, Y]=\Cov[X, a X+b]=a \Var[X], \quad \sigma_Y=|a| \sigma_X \\
	\]
	so
	\[
		\boxed{\rho=\frac{a \Var[X]}{\sigma_X|a| \sigma_X}=\fn{sgn}[a] \in \{\pm 1\}}
	\]
	(If \(a=0\), then \(\sigma_Y=0\) and \(\rho\) is undefined.)

\end{hwkProblem}

\begin{hwkProblem}{3}{Probability Density} \label{hwk:p03}

	Consider the following function
	\[
		\fn*{f_{XY}}[x, y] = \begin{cases}a e^{-2 x} e^{-3 y} & x>0, y>0 \\ 0 & \text {otherwise}\end{cases}
	\]
	and answer the following questions.

	\begin{enumerate}[label=(\alph*)]
		\item \label{hwk:p03a} Find the value of \(a\) so that \(f_{X Y}(x, y)\) is a valid joint probability density function.
		\item \label{hwk:p03b} Calculate \(\bar{x}\) and \(\bar{y}\).
		\item \label{hwk:p03c} Calculate \(\mathbb{E}\left[X^2\right], \mathbb{E}\left[Y^2\right]\), and \(\mathbb{E}[X Y]\).
		\item \label{hwk:p03d} Calculate the autocorrelation matrix of the random vector \(Z=[X, Y]^T\).
		\item \label{hwk:p03e} Calculate the variance \(\sigma_x^2\), variance \(\sigma_y^2\), and the covariance \(C_{X Y}\).
		\item \label{hwk:p03f} Calculate the autocovariance matrix of the random vector \(Z=[X, Y]^T\).
		\item \label{hwk:p03g} Calculate the correlation coefficient between \(X\) and \(Y\).
	\end{enumerate}

	\hwkSol{} \label{hwk:s03}

	\hwkPart{} \label{hwk:s03a}

	\[
		1=\iint f_{X Y}=a\left(\int_0^{\infty} e^{-2 x} d x\right)\left(\int_0^{\infty} e^{-3 y} d y\right)=a\left(\frac{1}{2}\right)\left(\frac{1}{3}\right)=\frac{a}{6} \Rightarrow a=6 .
	\]
	Thus \(\fn*{f_{X Y}}[x, y]=6 e^{-2 x} e^{-3 y}\) and factorizes:
	\[
		\boxed{X \sim \fn{Exp}[2], Y \sim \fn{Exp}[3]} \qquad \text{independent}
	\]

	\hwkPart{} \label{hwk:s03b}

	\[
		\boxed{\bar{x}=\mathbb{E}[X]=\frac{1}{2}, \quad \bar{y}=\mathbb{E}[Y]=\frac{1}{3}}
	\]

	\hwkPart{} \label{hwk:s03c}

	For \(\operatorname{Exp}(\lambda) \mathbb{E}\left[X^2\right]=\frac{2}{\lambda^2}\)
	\begin{align*}
		\mathbb{E}\left[X^2\right]&=\frac{2}{2^2}=\frac{1}{2} \\
		\mathbb{E}\left[Y^2\right]&=\frac{2}{3^2}=\frac{2}{9} \\
		\mathbb{E}[X Y]&=\mathbb{E}[X] \mathbb{E}[Y]=\frac{1}{2} \cdot \frac{1}{3}
	\end{align*}
	\[
		\boxed{\mathbb{E}[X Y]= \frac{1}{6}}
	\]

	\pagebreak

	\hwkPart{} \label{hwk:s03d}

	\[
		R_Z=\left[\begin{array}{cc}
			\mathbb{E}\left[X^2\right] & \mathbb{E}[X Y] \\
			\mathbb{E}[Y X] & \mathbb{E}\left[Y^2\right]
		\end{array}\right]=
		\left[\begin{array}{cc}
			\frac{1}{2} & \frac{1}{6} \\
			\frac{1}{6} & \frac{2}{9}
		\end{array}\right]
	\]

	\hwkPart{} \label{hwk:s03e}

	\[
		\sigma_x^2=\mathbb{E}\left[X^2\right]-\bar{x}^2=\frac{1}{2}-\frac{1}{4}=\frac{1}{4}
	\]
	\[
		\sigma_y^2=\frac{2}{9}-\frac{1}{9}=\frac{1}{9}
	\]
	\[
		C_{X Y}=\mathbb{E}[X Y]-\bar{x} \bar{y}=\frac{1}{6}-\frac{1}{6}=0
	\]

	\hwkPart{} \label{hwk:s03f}

	\[
		K_Z=\left[\begin{array}{cc}
			\sigma_x^2 & C_{X Y} \\
			C_{X Y} & \sigma_y^2
		\end{array}\right]=
		\left[\begin{array}{cc}
			\frac{1}{4} & 0 \\
			0 & \frac{1}{9}
		\end{array}\right]
	\]

	\hwkPart{} \label{hwk:s03g}

	\[
		\rho=\frac{C_{X Y}}{\sigma_x \sigma_y}=0
	\]

\end{hwkProblem}

\begin{hwkProblem}{4}{Covariance and Variance} \label{hwk:p04}

	Prove the following two results from lecture where \(x \sim \mathcal{N}\left(\bar{x}, \sigma_x^2\right)\) and \(e \sim \mathcal{N}\left(0, \sigma_e^2\right)\) and \(y=c x+d e\).

	\begin{enumerate}[label=(\alph*)]
		\item \label{hwk:p04a} \(\Cov[X, Y]=\mathbb{E}[(x-\bar{x})(y-\bar{y})]=\mathbb{E}[X Y]-\bar{x} \bar{y}\)
		\item \label{hwk:p04b} \(\Var[Y]=\mathbb{E}\left[(y-\bar{y})^2\right]=c^2 \sigma_x^2+d^2 \sigma_e^2\)
	\end{enumerate}
	\hwkSol{} \label{hwk:s04}

	\hwkPart{} \label{hwk:s04a}

	Let \(\mu_X=\mathbb{E}[X]\) and \(\mu_Y=\mathbb{E}[Y]\).
	\[
		\begin{aligned}
			\mathbb{E}\left[\left(X-\mu_X\right)\left(Y-\mu_Y\right)\right] & =\mathbb{E}\left[X Y-\mu_X Y-\mu_Y X+\mu_X \mu_Y\right] \\
											& =\mathbb{E}[X Y]-\mu_X \mathbb{E}[Y]-\mu_Y \mathbb{E}[X]+\mu_X \mu_Y \\
											& =\mathbb{E}[X Y]-\mu_X \mu_Y-\mu_Y \mu_X+\mu_X \mu_Y \\
											& =\mathbb{E}[X Y]-\mu_X \mu_Y
		\end{aligned}
	\]
	Thus
	\[
		\Cov[X, Y]=\mathbb{E}\left[\left(X-\mu_X\right)\left(Y-\mu_Y\right)\right]=\mathbb{E}[X Y]-\mathbb{E}[X] \mathbb{E}[Y]
	\]
	\[
		\implies \boxed{\Cov[X, Y]=\mathbb{E}[X Y]-\bar{x}\bar{y}}
	\]

	\hwkPart{} \label{hwk:s04b}

	\[
		\bar{y}=\mathbb{E}[Y]=c \mathbb{E}[x]+d \mathbb{E}[e]=c \bar{x}
	\]
	Then
	\[
		Y-\bar{y}=c(x-\bar{x})+d e
	\]
	Hence
	\[
		\Var[Y]=\mathbb{E}\left[(c(x-\bar{x})+d e)^2\right]=c^2 \sigma_x^2+d^2 \sigma_e^2+2 c d \mathbb{E}[(x-\bar{x}) e]
	\]
	If \(x\) and \(e\) are independent (or just uncorrelated), \(\mathbb{E}[(x-\bar{x}) e]=0\), giving
	\[
		\boxed{\Var[Y]=c^2 \sigma_x^2+d^2 \sigma_e^2}
	\]

\end{hwkProblem}

\begin{hwkProblem}{5}{Central Limit Theorem + Mappings} \label{hwk:p05}

	\begin{enumerate}[label=(\alph*)]
		\item \label{hwk:p05a} In python, generate a random variable \(x_1\) distributed by a uniform distribution \(x_i \sim \mathcal{U}[-1,1]\) using \texttt{np.random.uniform} function. Sample \(N=10\) points from this distribution, plot those points as a histogram.
		\item \label{hwk:p05b} Compute the sample mean, \(\hat{\mu}\), and sample variance, \(\hat{\sigma}^2\), of a sample set using functions \texttt{np.mean} and \texttt{np.var} functions, and determine if the reported values match the analytic mean and variance, \(\mathbb{E}\left[x_1\right]\) and \(\mathbb{E}\left[\left(x_1-\mu_{x_1}\right)^2\right]\) respectively? Repeat using \(N=10^i\) samples where \(1 \leq i \leq 6\), reporting the sample mean. Report the values and explain what you observe.
		\item \label{hwk:p05c} Create three new random variables \(x_2, x_3, x_4\) in addition to \(x_1\), each also distributed from a uniform distribution \(\mathcal{U}[-1,1]\). Sample \(N=100,000\) values from \(x_1-x_4\), and use these independent variables to compute a new set of random variables \(y_1, y_2\), and \(y_3\) defined as
			\[
				\begin{aligned}
					y_1&=\frac{x_1+x_2}{2} \\
					y_2&=\frac{x_1+x_2+x_3}{3} \\
					y_3&=\frac{x_1+x_2+x_3+x_4}{4}
				\end{aligned}
			\]
			Using the sampled values of \(x_1\) through \(x_4\), plot a histogram \(\fn*{p}[x_{1}], \fn*{p}[y_{1}], \fn*{p}[y_{2}], \fn*{p}[y_{3}]\).
		\item \label{hwk:p05d} Explain what you see.
		\item \label{hwk:p05e} Transform \(y_3\) into two new random variables \(z\) and \(q\) and plot the resulting distributions of \(\fn*{p}[z]\) and \(\fn*{p}[q]\) next to the original distribution \(\fn*{p}[y_{3}]\) where
			\[
				\begin{aligned}
					z&=\fn*{g}[y]=2 y+3 \\
					q&=\fn*{f}[y]=e^y
				\end{aligned}
			\]
		\item \label{hwk:p05f} Explain how the distribution changes with both transformations, and if it makes sense.
	\end{enumerate}

	\hwkSol{} \label{hwk:s05}

	\hwkPart{} \label{hwk:s05a}

	\begin{figure}[H] \label{fig:s05a}
		\begin{center}
			\includegraphics[width=0.45\textwidth]{./outputs/figures/s05a.png}
		\end{center}
		\caption{Histogram of \( N = 10 \) Samples of \( x_{1} = \mathcal{U}[-1, 1] \)}
	\end{figure}

	\hwkPart{} \label{hwk:s05b}

	\inputminted[autogobble, firstline=2, firstnumber=1]{python}{./outputs/text/s05b.txt}

	\hwkPart{} \label{hwk:s05c}

	\begin{figure}[H] \label{fig:s05c}
		\begin{center}
			\begin{subfigure}{0.45\textwidth} \label{fig:s05c1}
				\includegraphics[width=\linewidth]{./outputs/figures/s05c1.png}
				\caption{\( \fn*{p}[x_{1}] \) where \( x_{1} = \mathcal{U}[-1, 1] \)}
			\end{subfigure}
			\hfill
			\begin{subfigure}{0.45\textwidth} \label{fig:s05c2}
				\includegraphics[width=\linewidth]{./outputs/figures/s05c2.png}
				\caption{\( \fn*{p}[y_{1}] \) where \( y_{1} = \frac{x_{1}+x_{2}}{2} \)}
			\end{subfigure}
			\\
			\begin{subfigure}{0.45\textwidth} \label{fig:s05c3}
				\includegraphics[width=\linewidth]{./outputs/figures/s05c3.png}
				\caption{\( \fn*{p}[y_{2}] \) where \( y_{2} = \frac{x_{1}+x_{2}+x_{3}}{3} \)}
			\end{subfigure}
			\hfill
			\begin{subfigure}{0.45\textwidth} \label{fig:s05c4}
				\includegraphics[width=\linewidth]{./outputs/figures/s05c4.png}
				\caption{\( \fn*{p}[y_{3}] \) where \( y_{3} = \frac{x_{1}+x_{2}+x_{3}+x_{4}}{4} \)}
			\end{subfigure}
		\end{center}
	\end{figure}

	\pagebreak

	\hwkPart{} \label{hwk:s05d}

	\inputminted[autogobble, firstline=2, firstnumber=1]{python}{./outputs/text/s05d.txt}

	\hwkPart{} \label{hwk:s05e}
	
	\begin{figure}[H] \label{fig:s05e}
		\begin{center}
			\begin{subfigure}{0.45\textwidth} \label{fig:s05e1}
				\includegraphics[width=\linewidth]{./outputs/figures/s05e1.png}
				\caption{\( \fn*{p}[y_{3}] \) where \( y_{3} = \frac{x_{1}+x_{2}+x_{3}+x_{4}}{4} \)}
			\end{subfigure}
			\\
			\begin{subfigure}{0.45\textwidth} \label{fig:s05e2}
				\includegraphics[width=\linewidth]{./outputs/figures/s05e2.png}
				\caption{\( \fn*{p}[z] \) where \( z = \fn*{g}[y]=2 y+3 \)}
			\end{subfigure}
			\hfill
			\begin{subfigure}{0.45\textwidth} \label{fig:s05e3}
				\includegraphics[width=\linewidth]{./outputs/figures/s05e3.png}
				\caption{\( \fn*{p}[q] \) where \( q = \fn*{f}[y]=e^y \)}
			\end{subfigure}
		\end{center}
	\end{figure}

	\hwkPart{} \label{hwk:s05f}

	\inputminted[autogobble, firstline=2, firstnumber=1]{python}{./outputs/text/s05f.txt}

	\hwkCode{} \label{code:s05}

	See the \href{https://www.github.com/vaisriv/enae441-hw04/blob/main/src/hw04.py}{Python code} for this assignment.

\end{hwkProblem}

\end{document}
